\section*{Abstract}

In a time when open innovation (OI) is a key factor in gaining a long-term competitive edge, small and medium-sized businesses (SMEs) in developing countries like Tanzania confront major organizational problems that make it hard for them to use external collaborative models. This doctoral thesis, titled "Navigating the Open Innovation Paradigm: An Analysis of Organizational Barriers and the Critical Moderating Influence of Digital Literacy in Tanzanian SMEs," rigorously examines the multifaceted resistance factors within Tanzanian SMEs, employing a mixed-methods approach to elucidate the interplay between internal impediments and the transformative potential of digital literacy as a moderator. Based on Chesbrough's open innovation framework and expanded through institutional and resource-based theories, the study integrates global literature on OI barriers—including rigid hierarchical structures, risk aversion, distrust of external partners, resource limitations, and cultural inertia—while situating their occurrence within Tanzania's SME environment. Tanzania's growing SME sector, which accounts for more than 40\% of jobs, faces challenges in infrastructure and policy, as shown by national baseline surveys and economic indicators. This makes it a unique case where traditional closed innovation models continue to exist even as digitalization speeds up.

Using a sequential explanatory mixed-methods methodology, the study combines quantitative survey data from 313 SME owners and managers in sectors such as manufacturing, retail, ICT, and agriculture with qualitative insights from in-depth interviews. Structural equation modeling (SEM) and moderation analyses, employing scales for organizational barriers (e.g., cultural resistance, resource inadequacy), digital literacy (operationalized across technical, informational, communicative, and strategic dimensions), and OI adoption outcomes, demonstrate a significant negative correlation between barriers and OI engagement ($\beta = -0.42$, $p < 0.001$). Digital literacy serves as a significant moderator ($\Delta R^2 = 0.18$, $p < 0.01$), mitigating this negative impact by improving knowledge retention, facilitating networking, and enhancing adaptability—especially in resource-limited contexts where elevated digital literacy levels are associated with a 25-30\% rise in collaborative innovation propensity.

Thematic analysis of interviews supports these results, emphasizing particular Tanzanian dynamics, including legislative challenges and infrastructure deficiencies, but also identifying discrepancies where qualitative narratives show the impact of generational digital divides. In contrast to worldwide benchmarks derived from datasets such as the World Bank's Enterprise Surveys, the research reveals distinct contextual amplifiers, including restricted access to digital tools amidst escalating consumer price index (CPI) and tourism-induced economic pressures.

Theoretically, this thesis enhances open innovation research by introducing a sophisticated conceptual model that incorporates digital literacy as a boundary-spanning moderator in emerging settings, contesting universalist assumptions and highlighting socio-technical circumstances. In practical terms, it provides concrete suggestions for Tanzanian policymakers—such as targeted digital training initiatives under the National ICT Policy—and SME leaders, promoting hybrid innovation techniques to enhance resilience. Recognized limitations, such as self-reported biases and cross-sectional data, are addressed, and suggestions for longitudinal and comparative study are presented. This paper ultimately reveals strategies for Tanzanian SMEs to overcome OI hurdles, using digital literacy to foster inclusive economic development in Sub-Saharan Africa.

\textbf{Keywords:} Key terms include open innovation, digital literacy, Tanzanian SMEs, organizational barriers, and collaborative innovation.