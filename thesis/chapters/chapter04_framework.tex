\chapter{Theoretical Framework and Conceptual Model}\label{ch:framework}

\section{Theoretical Lenses}
This thesis integrates the resource-based view (RBV), dynamic capabilities, and institutional theory to explain how internal resources and external constraints shape OI. Digital literacy is conceptualized as a cross-cutting organizational capability that enhances sensing, seizing, and transforming activities.

\section{Conceptual Model and Hypotheses}
We posit that organizational barriers negatively affect OI engagement and that digital literacy weakens this negative relationship. Control variables include firm size, age, sector, and owner education.

\begin{figure}[H]
  \centering
  \includegraphics[width=0.85\textwidth]{conceptual_model.png}
  \caption{Conceptual model depicting barriers, digital literacy as moderator, and OI engagement.}
  \label{fig:conceptual_model}
\end{figure}

\section{Operationalization}
Barriers are measured across structural rigidity, risk aversion, trust deficits, and resource constraints. Digital literacy spans technical, informational, communicative, and strategic dimensions. OI engagement comprises inbound, outbound, and coupled practices.
