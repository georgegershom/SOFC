\chapter{Literature Review on Open Innovation and Organizational Barriers}\label{ch:literature}

\section{Evolution of Open Innovation}
\subsection{Chesbrough's Paradigm and Foundational Concepts}
The open innovation paradigm posits purposive inflows and outflows of knowledge to accelerate internal innovation and expand markets for external use of innovation.

\subsection{Theoretical Maturation and Global Trends}
Subsequent scholarship integrates dynamic capabilities, absorptive capacity, and platform ecosystems, extending OI across industries and firm sizes.

\subsection{Global Applications in SMEs}
SMEs utilize OI for tapping external R\&D, user innovation, and inter-firm collaboration, albeit with constraints distinct from large firms.

\section{Organizational Barriers to OI}
\subsection{Rigid Structures and Hierarchical Inertia}
Centralized decision-making and rigid routines impede external knowledge integration.

\subsection{Risk Aversion and Uncertainty Avoidance}
Perceived IP leakage, failure stigma, and resource risk constrain openness.

\subsection{Trust Deficits and Relational Barriers}
Weak inter-organizational trust and lack of governance mechanisms suppress collaborations.

\section{Mitigation Strategies for Barriers}
\subsection{R\&D Investment as a Barrier Mitigator}
Targeted R\&D, boundary spanners, and formalized partner selection can reduce frictions.

\section{OI in Emerging Economies and SMEs}
\subsection{Global vs. Emerging Economy Adaptations}
Institutional voids and infrastructure deficits reshape OI practices and partner portfolios.

\subsection{SME-Specific Adaptations in LDCs}
Resource scarcity increases reliance on digital platforms, public research organizations, and diaspora networks.

\section{Gaps in Existing Literature}
\subsection{Under-Representation of African Contexts}
Empirical studies on African SMEs remain sparse, especially on digital literacy as moderator.
