\chapter{Introduction}
\section{Background to the Study}
Open innovation (OI) posits purposive knowledge flows across organizational boundaries to accelerate internal innovation and expand external commercialization opportunities \parencite{chesbrough2003,west_bogers_2014}. In emerging economies, SMEs face layered institutional and resource constraints shaping their OI engagement.

\section{Problem Statement}
Despite Tanzania's expanding SME sector, organizational barriers---including hierarchical rigidity, risk aversion, and trust deficits---limit OI adoption.

\section{Research Objectives}
\subsection{General Objective}
To examine how organizational barriers impede OI in Tanzanian SMEs and how digital literacy moderates this relationship.
\subsection{Specific Objectives}
(1) Identify salient organizational barriers; (2) Assess direct effects on OI; (3) Evaluate moderating influence of digital literacy; (4) Compare patterns across sectors and firm sizes; (5) Derive policy and managerial implications.

\section{Research Questions}
Main: How do organizational barriers influence OI adoption in Tanzanian SMEs, and how does digital literacy moderate this effect? Specific questions align with objectives.

\section{Significance of the Study}
The study advances OI theory by contextualizing digital literacy as a boundary-spanning moderator and informs policy under the National ICT Policy \parencite{tz_national_ict_policy}.

\section{Scope and Delimitations}
Focus on SMEs in manufacturing, retail, ICT, and agriculture; cross-sectional design; synthetic yet calibrated dataset used for methodological illustration.

\section{Definition of Key Terms}
Open innovation, digital literacy, organizational barriers, SMEs.

\section{Thesis Structure}
Nine chapters covering context, literature, theory, methods, quantitative and qualitative results, discussion, and conclusions.
